\documentclass{article}
\usepackage[utf8]{inputenc}

\begin{document}

\title{Does Compulsory School Attendance Affect Schooling and Earnings?}
\author{Joshua D. Angrist and Alan B. Krueger}
\date{Quarterly Journal of Economics, Vol. 106, No. 4, Nov. 1991, pp. 979-1014}

\maketitle

\section*{Summary}
This seminal paper introduces an innovative approach to estimate the causal effect of education on earnings by leveraging the quarter of birth as an instrumental variable. The authors utilize compulsory schooling laws and fixed school entry age to address endogeneity issues in the education-earnings relationship, allowing for a clearer causal interpretation.

\section*{Key Points of the Study}

\begin{itemize} \scriptsize
\item \textbf{Objective:} Examine the impact of compulsory school attendance on schooling levels and subsequent earnings.
\item \textbf{Methodology \& Instrument:} Use of the quarter of birth as an instrumental variable for education, exploiting variations in educational attainment induced by compulsory schooling laws.
\item \textbf{Reason:} The approach addresses endogeneity issues by isolating the impact of education on earnings from other potential biases.
\item \textbf{Data:} Analysis based on U.S. Census data, leveraging public records to achieve a robust estimation of the education-earnings relationship.
\item \textbf{Results:} The study finds a positive causal impact of education on earnings, highlighting the significance of compulsory education policies.
\end{itemize}

\section*{Pedagogical and Replicability Value}
The paper serves as an excellent example of using instrumental variables to tackle endogeneity in empirical economics, aiding in the understanding of valid instrument criteria. Additionally, its reliance on publicly available data and clear documentation facilitates classroom replication exercises, providing valuable practical experience in IV estimation.

\end{document}